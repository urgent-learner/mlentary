\begin{document}
This figure demonstrates how the output image of the convolution operation is formed. 
First, the function selects particular elements defined by a window of the same size as the kernel.
Then the algorithm applies the weighted sum to this window and kernel;
and stores the result according to the window's position. 
The algorithm starts at the "top left" corner of the image, 
i. e. an element of the window at $(1, 1)$ is an element of the image at $(1, 1)$; 
and the output element is also at $(1, 1)$.
To compute the next element of the output image, the algorithm shifts the window by one.
For instance, if we want to compute an output element $(2, 1)$ we shift the window by one row and 
the window starts from 
$(2, 1)$ element of the input image.
Considering an input image of size $(W, H)$ and the kernel of size $(K, K)$, 
the overall computational complexity of 2D convolution is $O(W \times H \times K^2)$.
\end{document}